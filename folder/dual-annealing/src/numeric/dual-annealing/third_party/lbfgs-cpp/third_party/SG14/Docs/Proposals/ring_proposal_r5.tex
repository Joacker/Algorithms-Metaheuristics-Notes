Document number: P0059R4
Date: 2017-05-15
Reply-to: Guy Davidson, guy@hatcat.com
Reply-to: Arthur O’Dwyer, arthur.j.odwyer@gmail.com
Audience: Library Evolution (LEWG), Game dev and low latency (SG14)
A proposal to add a ring span to the standard library
0. Contents
Introduction
Motivation
Impact on the standard
Design decisions
Header <ring_span> synopsis
5.1 Function specifications: *_popper
5.2 Function specifications: ring_span
Sample use
Future work
Acknowledgements
1. Introduction
This proposal introduces a ring to the standard library operating on a span, named ring_span.  The ring_span offers similar facilities to std::queue with the additional feature of storing the elements in contiguous memory and being of a fixed size.  It is an update to P0059R2 to withdraw the addition of a concurrent ring span, at the request of SG1, and to remove iterator semantics, which are considered inappropriate for containers of unowned objects, at the request of LEWG.  The authors seek feedback on the design of the ring before submitting wording for the standard.
2. Motivation
Queues are widely used containers for collecting data prior to processing in order of entry to the queue (first in, first out).  The std::queue container adaptor acts as a wrapper to an underlying container, typically std::deque or std::list.  These containers are not of a fixed size and may grow as they fill, which means that each item that is added to a std::queue may prompt an allocation, which will lead to memory fragmentation.  The ring_span operates on elements in contiguous non-owned memory, so memory allocation is eliminated.  The most common uses for the ring_span would be:
Storing the last n events for later recovery
Communicating between threads in an allocation-constrained environment
Both of these use cases demand a single producer and a single consumer of elements.
3. Impact on the standard
This proposal is a pure library extension.  It does not require changes to any standard classes, functions or headers.
4. Design decisions
Naming
In an earlier version of this paper the name ring_buffer was proposed, but given the new implementation the proposed name is ring_span.  The name ring still remains the preferred choice of the authors.
Look like std::queue
There is already an object that offers FIFO support: the std::queue container.  The queue grows to accommodate new entries, allocating new memory as necessary.  The ring interface can therefore be similar to that of the queue with the addition of try_push, try_emplace and try_pop functions: these must now fail if they are called when the ring is full (or empty in the case of try_pop), and should therefore signal that condition by returning a success/fail value.
push_back and pop_front
Pushing items is a simple matter of assigning to a pre-existing element. The user can decide what to do on filling up the ring: for the synchronous ring it is possible to overwrite unpopped items, which would be desirable for the use case of storing the last n events for later recovery.
 
Popping items is a more complicated matter than in other containers.  If an item is popped from a std::queue it is destroyed and the memory is released.  In the case of a ring_span however, it does not own the memory so a different strategy must be pursued.  There are four things that could happen when an object is popped from a ring_span, besides the usual container housekeeping:
The object is destroyed via the class destructor and the memory is left in an undefined state.
The object is replaced with a default-constructed object.
The object is replaced with a copy of a user-specified object.
The object is not replaced at all.
 
This is a choice that will depend on the type being contained.  For example, if the type is not default-constructible, option 2 is unavailable.  If the type is not assignable, options 2 and 3 are unavailable.  There is no single solution that covers all these situations, so as part of the definition of ring_span a number of pop strategy objects are defined.  A strategy can be chosen at the point of declaration of an instance of a ring_span as a template parameter.
 
Although pop could theoretically safely be called on an empty ring with the implementation supplied below, it should yield undefined behaviour.
5. Header <ring_span> synopsis
This section contains the header declarations.  Example definitions are also provided for clarity and to aid specification of the definitions.
namespace std::experimental {
template <typename T> struct null_popper
{
  void operator()(T&) const noexcept;
};
 
template <typename T> struct default_popper
{
  T operator()(T& t) const;
};
 
template <typename T> struct copy_popper
{
  explicit copy_popper(T&& t);
  T operator()(T& t) const;
  T copy;
};
 
template<typename T, class Popper = default_popper<T>>
class ring_span
{
public:
  using type = ring_span<T, Popper>;
  using size_type = std::size_t;
  using value_type = T;
  using pointer = T*;
  using reference = T&;
  using const_reference = const T&;
 
  template <class ContiguousIterator>
  ring_span(ContiguousIterator begin, ContiguousIterator end,
            Popper p = Popper()) noexcept;
 
  template <class ContiguousIterator>
  ring_span(ContiguousIterator begin, ContiguousIterator end,
            ContiguousIterator first, size_type size,
            Popper p = Popper()) noexcept;
 
  ring_span(ring_span&&) = default;
  ring_span& operator=(ring_span&&) = default;
 
  bool empty() const noexcept;
  bool full() const noexcept;
  size_type size() const noexcept;
  size_type capacity() const noexcept;
 
  reference front() noexcept;
  const_reference front() const noexcept;
  reference back() noexcept;
  const_reference back() const noexcept;
 
  template<bool b = true,
           typename = std::enable_if_t<b &&
                      std::is_copy_assignable<T>::value>>
  void push_back(const value_type& from_value) 
           noexcept(std::is_nothrow_copy_assignable<T>::value);
 
  template<bool b = true,
           typename = std::enable_if_t<b &&
                      std::is_move_assignable<T>::value>>
  void push_back(value_type&& from_value) 
           noexcept(std::is_nothrow_move_assignable<T>::value);
 
  template<class... FromType>
  void emplace_back(FromType&&... from_value) 
         noexcept(std::is_nothrow_constructible<T, FromType...>::value && 
                  std::is_nothrow_move_assignable<T>::value);
 
  T pop_front();
 
  void swap(type& rhs)
           noexcept (std::is_nothrow_swappable<Popper>::value);
 
// Example implementation
private:
  reference at(size_type idx) noexcept;
  const_reference at(size_type idx) const noexcept;
  size_type back_idx() const noexcept;
  void increase_size() noexcept;
 
  T* m_data;
  size_type m_size;
  size_type m_capacity;
  size_type m_front_idx;
  Popper m_popper;
};
5.1. Function specifications: *_popper
The null_popper object does nothing to the item being popped from the ring.
 template <typename T>
 void null_popper::operator()(T&) const noexcept
 {};
 
The default_popper object moves the item being popped from the ring into the return value.
 template <typename T>
 T default_popper::operator()(T& t) const
 {
  return std:move(t);
 }
 
The copy_popper object replaces the item being popped from the ring with a copy of an item of the contained type, chosen at the declaration site.
 template <typename T>
 explicit copy_popper::copy_popper(T&& t)
  : copy(std::move(t))
 {}
 
 template <typename T>
 T copy_popper::operator()(T& t) const
 {
  T old = std::move(t);
  t = copy;
  return old;
 }
5.2 Function specifications: ring_span
The first constructor takes a range delimited by two contiguous iterators and an instance of a popper.  After this constructor is executed, the capacity of the ring is the distance between the two iterators and the size of the ring is its capacity.  A typical implementation would be
template<typename T, class Popper>
template<class ContiguousIterator>
ring_span<T, Popper>::ring_span(ContiguousIterator begin, ContiguousIterator end, Popper p) noexcept
  : m_data(&*begin)
  , m_size(0)
  , m_capacity(end - begin)
  , m_front_idx(0)
  , m_popper(std::move(p))
  {}
 
The second constructor creates a partially full ring.  It takes a range delimited by two contiguous iterators, a third iterator which points to the oldest item of the ring, a size parameter which indicates how many items are in the ring, and an instance of a popper.  After this constructor is executed, the capacity of the ring is the distance between the first two iterators and the size of the ring is the size parameter.  A typical implementation would be
template<typename T, class Popper>
template<class ContiguousIterator>
ring_span<T, Popper>::ring_span(ContiguousIterator begin, ContiguousIterator end, ContiguousIterator first, size_type size, Popper p = Popper()) noexcept
  : m_data(&*begin)
  , m_size(size)
  , m_capacity(end - begin)
  , m_front_idx(first - begin)
  , m_popper(std::move(p))
{}
 
empty(), full(), size() and capacity() behave as expected.  Typical implementations would be:
template<typename T, class Popper>
bool ring_span<T, Popper>::empty() const noexcept
{ return m_size == 0; }
template<typename T, class Popper>
bool ring_span<T, Popper>::full() const noexcept
{ return m_size == m_capacity; }
template<typename T, class Popper>
ring_span<T, Popper>::size_type ring_span<T, Popper>::size() const noexcept
{ return m_size; }
template<typename T, class Popper>
ring_span<T, Popper>::size_type ring_span<T, Popper>::capacity() const noexcept
{ return m_capacity; }
 
front() and back() return the oldest and newest items in the ring.  Typical implementations would be:
template<typename T, class Popper>
ring_span<T, Popper>::reference ring_span<T, Popper>::front() noexcept
{ return *begin(); }
template<typename T, class Popper>
ring_span<T, Popper>::reference ring_span<T, Popper>::back() noexcept
{ return *(--end()); }
template<typename T, class Popper>
ring_span<T, Popper>::const_reference ring_span<T, Popper>::front() const noexcept
{ return *begin(); }
template<typename T, class Popper>
ring_span<T, Popper>::const_reference ring_span<T, Popper>::back() const noexcept
{ return *(--end()); }
 
The push_back() functions add an item after the most recently added item.  The emplace_back() function creates an item after the most recently added item.  If the size of the ring equals the capacity of the ring, then the oldest item is replaced.  Otherwise, the size of the ring is increased by one.  Typical implementations would be:
template<typename T, class Popper>
template<bool b=true, typename=std::enable_if_t<b && std::is_copy_assignable<T>::value>>
void ring_span<T, Popper>::push_back(const T& value) noexcept(std::is_nothrow_copy_assignable<T>::value)
{
  m_data[back_idx()] = value;
  increase_size();
}
 
template<typename T, class Popper>
template<bool b=true, typename=std::enable_if_t<b && std::is_move_assignable<T>::value>>
void ring_span<T, Popper>::push_back(T&& value) noexcept(std::is_nothrow_move_assignable<T>::value)
{
  m_data[back_idx()] = std::move(value);
  increase_size();
}
 
template<typename T, class Popper>
template<class... FromType>
void ring_span<T, Popper>::emplace_back(FromType&&... from_value)
noexcept(std::is_nothrow_constructible<T, FromType...>::value &&
         std::is_nothrow_move_assignable<T>::value);
{
  m_data[back_idx()] = T(std::forward<FromType>(from_value)...);
  increase_size();
}
 
The pop_front() function checks the size of the ring, asserting if it is zero.  If it is non-zero, it passes a reference to the oldest item to the Popper for transformation, reduces the size and advances the front of the ring.  By returning the item from pop, we are able to contain smart pointers.  A typical implementation might be:
template<typename T, class Popper>
auto ring_span<T, Popper>::pop_front()
{
  assert(m_size != 0);
  auto old_front_idx = m_front_idx;
  m_front_idx = (m_front_idx + 1) % m_capacity;
  --m_size;
  return m_popper(m_data[old_front_idx]);
}
 
The swap() function is trivial. A typical implementation might be:
template<typename T, class Popper>
void ring_span<T, Popper>::swap(ring_span<T, Popper>& rhs) noexcept(std::__is_nothrow_swappable<Popper>::value)
{
  using std::swap;
  swap(m_data, rhs.m_data);
  swap(m_size, rhs.m_size);
  swap(m_capacity, rhs.m_capacity);
  swap(m_front_idx, rhs.m_front_idx);
  swap(m_popper, rhs.m_popper);
}
 
For the sake of clarity, the private implementation used to describe these functions is as follows:
template<typename T, class Popper>
ring_span<T, Popper>::reference ring_span<T, Popper>::at(size_type i) noexcept
{ return m_data[i % m_capacity]; }
 
template<typename T, class Popper>
ring_span<T, Popper>::const_reference ring_span<T, Popper>::at(size_type i) const noexcept
{ return m_data[i % m_capacity]; }
 
template<typename T, class Popper>
ring_span<T, Popper>::size_type ring_span<T, Popper>::back_idx() const noexcept
{ return (m_front_idx + m_size) % m_capacity; }
 
template<typename T, class Popper>
void ring_span<T, Popper>::increase_size() noexcept
{ if (++m_size > m_capacity) { m_size = m_capacity; } }
6. Sample use
#include <ring_span>
#include <cassert>
 
using std::experimental::ring_span;
 
void ring_test()
{
  std::array<int, 5> A;
  std::array<int, 5> B;
 
  ring_span<int> Q(std::begin(A), std::end(A));
 
  Q.push_back(7);
  Q.push_back(3);
  assert(Q.size() == 2);
  assert(Q.front() == 7);
 
  Q.pop_front();
  assert(Q.size() == 1);
 
  Q.push_back(18);
  auto Q3 = std::move(Q);
  assert(Q3.front() == 3);
  assert(Q3.back() == 18);
 
  sg14::ring_span<int> Q5(std::move(Q3));
  assert(Q5.front() == 3);
  assert(Q5.back() == 18);
  assert(Q5.size() == 2);
 
  Q5.pop_front();
  Q5.pop_front();
  assert(Q5.empty());
 
  sg14::ring_span<int> Q6(std::begin(B), std::end(B));
  Q6.push_back(6);
  Q6.push_back(7);
  Q6.push_back(8);
  Q6.push_back(9);
  Q6.emplace_back(10);
  Q6.swap(Q5);
  assert(Q6.empty());
  assert(Q5.size() == 5);
  assert(Q5.front() == 6);
  assert(Q5.back() == 10);
 
  puts("Ring test completed.\n");
}
7. Future work
n3353 describes a proposal for a concurrent queue.  The interface is quite different from ring.  A concurrent ring could be adapted from the interface specified therein should n3353 be accepted into the standard.  Considerable demand has been expressed for such an entity by the embedded development community, but at the presentation of revision 2 of this paper to SG1 such a feature was turned down since it overlapped with n3353.  Feedback from developers in the embedded community suggests that a concurrent queue would not be used in their domain because of the contingent unpredictable memory allocation, and a fixed size container such as this would be preferable, even one with the size as a template parameter.
 
The popper class templates are defined at an overly broad scope, rather than in the scope of the ring_span.  However, no way of doing this is immediately apparent, beyond the obvious solution of creating a ring namespace and defining the poppers and the span inside it.  Since this is somewhat counterintuitive in the context of the remainder of the standard library, the authors remain open to suggestions.  If the popper class templates might have use in other container spans, then they could remain in the broader scope.
 
Requests have been made for a mechanism of providing notification when an item has been pushed. This could be achieved by creating a pusher policy, analogous to the popper objects. If this is deemed valuable then this proposal can be modified accordingly.
8. Acknowledgements
Thanks to Jonathan Wakely for sprucing up the first draft of the ring interface.
Thanks to the SG14 forum contributors: Nicolas Guillemot, John McFarlane, Scott Wardle, Chris Gascoyne, Matt Newport.
Thanks to the SG14 meeting contributors: Charles Beattie, Brittany Friedman, Billy Baker, Bob, Sean Middleditch, Ville Voutilainen.
Thanks to Michael McLaughlin for commentary on the draft of the text.
Thanks to Lawrence Crowl for pointing me to his paper on concurrent queues, n3353.
Special thanks also to Michael Wong for convening and shepherding SG14.
